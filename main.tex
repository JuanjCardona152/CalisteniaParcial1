\documentclass{article}
\usepackage[utf8]{inputenc}
\usepackage[spanish]{babel}
\usepackage{listings}
\usepackage{graphicx}
\graphicspath{ {images/} }
\usepackage{cite}

\begin{document}

\begin{titlepage}
    \begin{center}
        \vspace*{1cm}
            
        \Huge
        \textbf{Solución  al Desafío }
            
        \vspace{0.5cm}
        \LARGE
 
            
        \vspace{1.5cm}
            
        \textbf{Juan José Arboleda Cardona}
            
        \vfill
            
        \vspace{0.8cm}
            
        \Large
        Despartamento de Ingeniería Electrónica y Telecomunicaciones\\
        Universidad de Antioquia\\
        Medellín\\
        Marzo de 2021
            
    \end{center}
\end{titlepage}

\tableofcontents
\newpage
\section{Sección introductoria}\label{intro}
Con el fin de Preparar los Desafios que se enfrentaran en el curso se planteo un ejercicio y se puso a prueba la capacidad para dar instrucciones. En este caso se le daran unas instrucciones a 3 sujetos de prueba que no tienen conocimiento de lo que van a hacer, y sin ayuda, tendran que seguir al pie de la letra las instrucciones dadas.

\section{Desarollo del desafío} \label{contenido}
Con la hoja y las 2 tarjetas  a la mano, Siga al pie de la letra las siguientes instrucciones:
\begin{enumerate}
 \item con la hoja en vertical hacer 2 pliegues horizontales de manera que la hoja quede dividida en 3 partes iguales. 
 \item Abrir la hoja y darle la vuelta dejando que los pliegles sobresalgan dejando unas leves elevaciones en la parte de la hoja que quedo entre los 2 pliegues.
 \item Unir las tarjetas y de manera vertical cogerlas de la parte superior, la parte inferior de las tarjetas se debe ubicar en la parte de la hoja que esta entre los 2 pliegues. 
 \item ubicada, poner la parte inferior de las tarjetas paralelas a los pliegues y separarlas sin soltar la parte superior. 
 \item Acercar 1 la parte inferior de las tarjetas hacia el pliegue mas cercano y repetir con la otra parte inferior de la otra tarjeta.   
 \end{enumerate}
 \section{video} \label{video }
Anexo el link de Youtube para el video.
[espacio para el link]
\section {Conclusiones} \label{conclusion} 
Independiente de la capacidad de cada sujeto que se puso a prueba, se observa que, frente a indicaciones detalladas y bien dadas, se puede lograr un objetivo sin conocer el fin por el cual se realiza, asi como cuando le damos una instruccion a un computador cuando programamos.
\end{document}
